\documentclass[letterpaper,twoside,12pt]{article}
\usepackage[dvips]{graphicx}
\usepackage[margin=0.8in]{geometry}
\usepackage{caption}
\usepackage{amsmath}
\usepackage{amssymb}
\usepackage[T1]{fontenc}
\usepackage{natbib}
\usepackage{authblk}
%\usepackage{fancyvrb}
%\usepackage{lmodern}
%\usepackage{url}
%\usepackage{times}
\usepackage{textcomp}
%\usepackage{parskip}
\bibliographystyle{agu08}

\newcommand{\ud}{\mathrm d}
\newcommand{\uj}{\mathrm{j}}
\newcommand{\bz}{\mathbf{z}}
\newcommand{\Real}{\mathrm{Re}}
\newcommand{\Imag}{\mathrm{Im}}
\newcommand{\dif}{\mathrm{d}}
\newcommand{\sigsig}{\sigma_1\sigma_2}
\newcommand{\varss}{\varsigma_1\varsigma_2}
\newcommand{\hvarss}{\hat{\varsigma}_1 \hat{\varsigma}_2}
\newcommand{\twodots}{\mathinner {\ldotp \ldotp}}
\newcommand{\midtilde}{\raisebox{0.5ex}{\texttildelow}}

\DeclareMathOperator\erf{erf}

\title{Comparison of Linear and Circular Products after PolConvert}

\author[1]{L. V. Benkevitch}
\affil[1]{\small MIT Haystack observatory, Westford, MA 01886, USA.}


\begin{document}

\maketitle

\begin{abstract}

\end{abstract}


\tableofcontents


\section{Multi-Band Delays Before and After PolConvert}




\section{Convenience Software}

\subsection{make\_sorted\_idx.py: Saving VGOS data in Python dictionaries}

The VLBI Global Observing System (VGOS) database is organized as a tree-like directory structure. For our purpose of statistical analysis of a small number of parameters scattered across many directories and files below the root directory of the experiment, this implies significant overhead in opening multiple files and accessing the parameters within each of them. The data files in their names only provide the station or baseline names, and no time or polarization information. For example, extraction of, say, SNR data for a particular polarization product and within a specific time range would require opening \emph{all} the files and accessing their times and polarizations using HOPS API calls.

We wrote a script, \verb@make_sorted_idx.py@, to extract the parameters for statistical analysis for the whole experiment and to put it in a Python dictionary, preserving the temporal order. We call such dictionaries ``indices''. The index can be ``pickled'' and saved on disk. Interestingly, these files are small, in the hundreds of kilobytes.  The other data analisys and plotting scripts read the index files, unpickle them into the Python dictionaries, and use data from the dictionaries.

The script \verb@make_sorted_idx.py@ should be run on the \verb@demi.haystack.mit.edu@ server where the VGOS data are stored under the directory \verb@/data-sc16/geodesy/@. Current version of the script works on the experiment 3819 data located under \verb@/data-sc16/geodesy/3819@. It creates three dictionaries pickled in the files \verb@idx3819l.pkl@, \verb@idx3819c.pkl@, and \verb@idx3819cI.pkl@ in the directory where the script was run. 

\begin{itemize}
  \item \verb@idx3819l.pkl@: linear polarization products immediately from the \\
  \verb@/data-sc16/geodesy/3819/@ directory;
  
  \item \verb@idx3819c.pkl@: circular polarization products generated by PolConvertwithout the pseudo-Stokes
      \verb@'I'@ data, only \verb@'LL', 'LR', 'RL', 'RR'@. \\
  The data are found in \\
  \verb@/data-sc16/geodesy/3819/polconvert/3819/scratch/pol_prods1/3819@ directory.

  \item \verb@idx3819cI.pkl@: pseudo-Stokes 'I' only for the circular polarization products generated by
  PolConvert. The data are taken from \\
  \verb@/data-sc16/geodesy/3819/polconvert/3819/scratch/pcphase_stokes_test/3819@ \\
\end{itemize}

A pickle file can be unpickled into a  dictionary using the \verb@pickle.load()@ function. For example: \\ \\
\noindent \verb@import pickle@ \\
\noindent \verb@with open('idx3819c.pkl', 'rb') as finp:@ \\
\noindent \verb@    idx3819c_1 = pickle.load(finp)@ \\ \\

\noindent The script \verb@make_sorted_idx.py@ is also a Python module defining the function \\ \\
\verb@make_idx(base_dir, pol='lin', max_depth=2)@ with parameters: \\ \\
\verb@    base_dir@: the directory containing the VGOS data. For example, it may be \\
\verb@              /data-sc16/geodesy/3819/@. \\
\verb@    pol@: polarization, 'lin' - linear, 'cir' - circular. \\
    \indent \indent \indent This parameter is used for the data generated by PolConvert. \\
    \indent \indent \indent It converts the polarization product names\\
    \verb@          'XX', 'XY', 'YX', 'YY'@, and the lists \verb@['XX', 'YY']@ into the correct names \\
    \verb@          'LL', 'LR', 'RL', 'RR'@, and \verb@'I'@, respectively. \\
\verb@    max_depth@: Limits the maximum depth of recursing into the subdirectories of \verb@base_dir@. \\

\noindent \verb@make_idx()@ creates and returns the index dictionary with the data from \verb@base_dir@. \\

The index dictionary has three dimensions: the baseline name, the polarization, and the data proper, including 'time', 'file', 'mbdelay', 'sbdelay', and 'snr'. Consider a particular index named \verb@idx3819l_1@ (experiment 3819, linear polarization). Its first dimension is indexed with the baseline names derived from the set of stations, \verb@{'E', 'M', 'S', 'T', 'V', 'Y'}@. \\
\noindent The possible first indices are the baseline names: \\
\verb@idx3819l_1.keys()@ \\
\verb@dict_keys(['SE', 'VY', 'MV', 'MT', 'TV', 'EY', 'SY', 'TY', 'MS', 'SV',@ \\
\verb@           'TE', 'EV', 'MY', 'ME'])@ \\ \\
\noindent Each of the baselines is associated with the cross-corellation products and the pseudo-Stokes I parameter. Thus the second index is one of the products. For example, for the \verb@'ME'@ baseline: \\
\verb@idx3819l_1['ME'].keys()@ \\
\verb@dict_keys(['XX', 'XY', 'YX', 'YY', 'I'])@ \\ \\
For example, the times, the full data file names, SNRs, multi- and single-band delays for the 'SY' baseline and the 'XY' polarization products from this baseline are contained in the index dictionary under \\
\verb@idx3819l_1['SY']['XY']@: \\ \\
\verb@idx3819l_1['SY']['XY'].keys()@ prints \\
\verb@dict_keys(['time', 'file', 'mbdelay', 'sbdelay', 'snr'])@. \\ \\
For example, in order to access the multi-band delay data list in the ascending temporal order for the baseline \verb@'SV'@ and the pseudo-Stokes I, one should issue the following command: \\
\verb@mbd = idx3819l_1['SV']['I']['mbdelay'].@

\section{Conclusion}


\newpage



\end{document}



